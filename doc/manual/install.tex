%% Installation guide.

\section{Installing \mntr}
\subsection{System Requirements}
\mntr{} has been primarily developed on Debian and Ubuntu Linux operating
systems. It has been tested on {\tt i686} and {\tt x86\_64} architecutures of
the above operating systems. It is also regularly tested on MAC-OSX on both
architectures. For the rest of the manual, we will assume that a user or
developer is working on a computer with Linux operating system and all the
standard command line utilities. We provide MAC-OSX specific
instructions where important differences arise.


\subsection{Obtaining \mntr}
\mntr{} source code is available for download at the 
\href{http://wiki.mcs.anl.gov/minotaur/index.php/Minotaur\_Download}{\mntr-download
page}.

\subsection{Prerequisites}
The following software is necessary for compiling Minotaur.
\begin{enumerate}
\item Boost headers.
\item C++ compiler.
\item lapack and blas libraries. Header files are not required.
On Debian/Ubuntu: \code+apt-get install liblapack-dev libblas-dev+.
\end{enumerate}
Optional packages that are recommended:
\begin{enumerate}
\item Fortran compiler. 
On Debian/Ubuntu: \code+apt-get install gfortran+.
On Mac, they are available by default.
\item AMPL Solver Library.
\item Filter-SQP
\item Ipopt
\item Osi-CLP
\item CppUnit
\item CMake
\end{enumerate}

\subsection{Installing Prerequisites}
\subsubsection{Boost Headers}
\subsubsection{AMPL Solver Library}
\subsubsection{Filter-SQP}
\subsubsection{Ipopt}
\subsubsection{OSI-CLP}
\subsubsection{CppUnit}
\subsubsection{CMake}
On Debian/Ubuntu: \code+apt-get install cmake ccmake+. On Mac, download and
install from the CMake website \url{http://www.cmake.org/cmake/resources/software.html}.
\subsection{Installing Minotaur}
We will call the topmost directory that contains Minotaur source code the
MINOTAUR\_SRC directory. It has files like `CMakeLists.txt', `README', and
directories like `scripts', `examples', `doc' and `src'. Minotaur should never
be compiled in this directory. A different directory should be created and all
compiling should be done there. If a compiling process goes wrong or if the
user does not want a particular compilation, this directory can be removed
without affecting the Minotaur source code. 

Minotaur can be compiled in two ways: by using CMake, or manually.
Using CMake is preferred over doing it manually because CMake
automatically configures a few things:
\begin{itemize}
\item The fortran compiler is used and how the function names are mangled.
\item If the function `rusage()' is available for measuring time.
\end{itemize}
A user who has experience in using UNIX-style makefiles can also compile
Minotaur using a makefile that comes with the distribution.

\subsubsection{Compiling with CMake}
\label{sec:compile_cmake}
Suppose, we want to compile Minotaur in a directory called `build'. Follow the
following steps:
\begin{codeenv}
mkdir build
cd build
ccmake /path/to/MINOTAUR_SRC
\end{codeenv}
The path argument could be relative or absolute. Now we should see a dialog
box. Press `c' to configure. If CMake cannot find Boost headers installed in
your system, it will complain. Press `e' in this case. Enter the location in
the next dialog. When all options are set, the user will find a `g' option at
the bottom. Pressing `g' will complete the configuration and generate a
makefile. The following options can be set:
\begin{description}
\item[ASL\_DIR] The directory where both `asl.h' and `amplsolver.a' are
located.
\item[BOOST\_INC\_DIR] Directory where header files of Boost are located.
Not required if installed in the system and no error reported by CMake.
\item[BUILD\_SHARED\_LIBS] `ON' means shared libraries will be created. `OFF'
means static libraries are created. On MAC, this flag should be `OFF'.
\item[CMAKE\_BUILD\_TYPE] Set to debug, if debugging. Otherwise leave blank.
\item[CppUNIT\_INC\_DIR] Directory where header files of CppUnit are located.
The directory must conatin the file `cppunit/TestCase.h'.
\item[CPPUNIT\_LIB\_DIR] Directory where libraries files of CppUnit are located.
\item[FILTER\_SQP\_LIB] The full path to the FilterSQP library file. \\
e.g. /home/bob/filter/src/filterlib.a
\item[IPOPT\_INC\_DIR] Directory where header files of Ipopt are located.
The directory must conatin the file `coin/IpIpoptApplication.hpp'.
\item[IPOPT\_LIB\_DIR] Directory where libraries files of Ipopt are located.
\item[OSI\_INC\_DIR] Directory where header files of CppUnit are located.
The directory must conatin the file `cppunit/TestCase.h'
\item[OSI\_LIB\_DIR] Directory where libraries files of OsiClp are located.
\item[MNTR\_INSTALL\_PREFIX] Directory where libraries and executables of Minotaur
will be installed on doing `make install'. The deafult setting of `.' installs
them in the `lib' and 'bin' folders in the current directory.
\end{description}
Once all the necessary and desired options have been set, again hit `c'. CMake
will perform some checks on the inputs. If no errors are found, hit `g' and
CMake will exit after generating makefiles. If errors were found, hit `e' and
fix the appropriate options. Repeat until all errors have been fixed. Pressing
`q' will stop the CMake dialog without creating any makefiles. It may be used
if the user wants to stop the process.

If a makefile is generated successfully, the user can then do 
\begin{codeenv}
make 
\end{codeenv}
Minotaur will be compiled. If no errors are reported, perform some unit tests
by doing:
\begin{codeenv}
make test
\end{codeenv}
The above tests will work only if CppUnit was configured. Now install Minotaur
by doing:
\begin{codeenv}
make install
\end{codeenv}
The libraries and executables will be
compiled depending upon the inputs given to CMake. So, for example, the user
does not provide a filename in FILTER\_SQP\_LIB, both mntrfiltersqp and
mntrbqpd will not be compiled. If the user did not provide ASL\_DIR, on
executables will be generated. If there are errors in the above command, then
the user must rerun the CMake command and repeat.

If the \code+make install+ step completed successfully, then the user will
find libraries in the `lib' directory and executables in the `bin' directory.
The latter directory is created only if some executables need to be installed.
Another directory `include' is created. It has a `minotaur' subdirectory that
contains header files useful for developing code with Minotaur.
`bin', `include' and `lib' directories are generated in the
MNTR\_INSTALL\_PREFIX directory. It defaults to the same directory
where Minotaur is compiled (and never to `MINOTAUR\_SRC' directory).

\subsubsection{Compiling with CMake on command line}
If the user does not want to use the dialog box, CMake can also be used
non-interactively by calling `cmake' and passing it all the desired values.
Here is an example, that can be modified based on the user's system:

\begin{codeenv}
cmake                                                                         \ 
  -DBOOST_INC_DIR:PATH=/sandbox/minotaur-externals                            \
  -DASL_DIR:PATH=/sandbox/minotaur-externals/asl                              \
  -DFILTER_SQP_LIB:PATH=/sandbox/minotaur-externals/filter/src/filterlib.a    \
  -DIPOPT_INC_DIR:PATH=/sandbox/minotaur-externals/ipopt-3.8.3/build/include/ \
  -DIPOPT_LIB_DIR:PATH=/sandbox/minotaur-externals/ipopt-3.8.3/build/lib/     \
  -DOSI_INC_DIR:PATH=/sandbox/minotaur-externals/osi-0.102.1/build/include/   \
  -DOSI_LIB_DIR:PATH=/sandbox/minotaur-externals/osi-0.102.1/build/lib/       \
  -DCPPUNIT_INC_DIR:PATH=/sandbox/minotaur-externals/cppunit/build/include/   \
  -DCPPUNIT_LIB_DIR:PATH=/sandbox/minotaur-externals/cppunit/build/lib        \
  -DBUILD_SHARED_LIBS:BOOL=1                                                  \
  /path/to/MINOTAUR_SRC
\end{codeenv}
Then follow the instructions as in Section~\ref{sec:compile_cmake} to run
\code+make+, \code+make install+ and \code+make test+.

\subsubsection{Compiling manually}
\subsection{Illegally Installing Minotaur with Externals}
Make sure you have a C++ compiler, a fortran compiler, blas and lapack.
\begin{codeenv}
cd scripts
./build_with_externals -d ./build -M .. -j NCPUS -b
\end{codeenv}
where NCPUS is the number of CPUs you want to use for compiling.
On MAC, you will need some additional flags:
\begin{codeenv}
cd scripts
./build_with_externals -d ./build -M .. -j NCPUS -b -C -O -s
\end{codeenv}

